\chapter{抽象空间}\label{chap:introduction}

\section{引言}

\section{赋范空间}
设$X$是数域$K$上的向量空间($K=R$或$C$),若对每个$x\in X$指定一个实数$||x||$,称为x的范数,其满足以下范数公理。
\indent
\begin{description}
	\item[($N_1$)] 齐次性     \quad\qquad $||ax||=|a|\cdot||x||\quad a\in K$
	\item[($N_2$)] 三角不等式 \quad $||x+y|| \leq ||x|| + ||y||$
	\item[($N_3$)] 正定性     \quad\qquad $||x||\geq 0,\quad ||x||=0 \leftrightarrow  ||x||=0$
\end{description}
则称X为K上的赋范向量空间,简称赋范空间。

\noindent 注:
\begin{description}
	\item[1] 如K=R,称为实赋范空间,如K=C,则称为复赋范空间。
    \item[2] description
    \begin{description}
    	\item[a.] 称$|x| = \sqrt{\sum_i |x_i|^2}$称为Euclid范数,常用于解释赋范空间的模型。
    	\item[b.] 一般范数无具体计算公式,其本质在于范数公理,正是舍弃了特殊的表达式,才得到了具有高度抽象性的赋范空间理论。
    	\item[c.] 一旦将赋范空间理论应用于某个特定空间,就必须选择适当的范数公式。
    \end{description}
\end{description}


\noindent 
例:{\heiti 有界函数空间$B(\Omega)$}

设$\Omega$是任一非空集合,$B(\Omega)$是定义在$\Omega$上的有界实(复)函数之全体。其显然是一实(复)向量空间。
任给$\mu\in B(\Omega)$,令
$$||\mu||_0 =  \sup_{x\in\Omega} |\mu(x)|$$
\noindent 验证:
\begin{description}
	\item[$1^0$] 齐次性
	  $$||a\mu||_0 =  \sup_{x\in\Omega} |a\mu(x)|$$
	\item[$2^0$] 三角不等式
%	\begin{split}
		
%	\end{split}
	\item[$3^0$] 正定性
\end{description}
[说明]
\begin{enumerate}
	\item 自然数集$ N $上的有界函数就是有界数列。因此,有界数列空间$ B(N) $是一赋范空间,通常记做$ \ell^{\infty} $。
	
\end{enumerate}



设$ X $是一个给定的赋范空间,借用几何术语,赋予抽象概念以某种直观形象。
\begin{enumerate}
	\item [(1)] $ X $中的点成为点或向量,向量x可解释为从零元0到点x的有向线段。
	\item [(2)] $ \Vert x\Vert $称为$ x $的长度,当$\Vert x\Vert =1$ 时,称$ x $为单位向量。
	\item [(3)] 称 $ \Vert x-y\Vert $为点$ x $与点$ y $之间的距离,也记做$ d(x,y) $。
	\item [(4)] 
\end{enumerate}

\section{Banach空间}

\section{常用函数空间}

\section{内积空间与Hilbert空间}
