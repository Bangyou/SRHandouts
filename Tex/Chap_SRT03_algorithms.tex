\chapter{稀疏复原问题的算法}\label{chap:srt03:algorithms}

\section{前言}
 
本章将对几种 常用的稀疏信号复原算法,如贪婪算法、有效集方法(如LARS算法)、块坐标下降、迭代阈值与近端法等进行总结。主要针对前面介绍的含噪稀疏复原问题,即最终不易处理的$ \ell_0 $范数最小化问题。
\begin{equation}\label{key}
(p_{0}^{\epsilon}):\qquad \min_{x}\Vert x\Vert_0\quad s.t.\quad\Vert y-Ax\Vert_2\leq \varepsilon
\end{equation}
与$ \ell_1 $范数松弛,也成为LASSO或基追踪
\begin{equation*}\label{key}
(p_{1}^{\epsilon}):\qquad \min_{x}\Vert x\Vert_1\quad s.t.\quad\Vert y-Ax\Vert_2^2\leq v
\end{equation*}
以及等价的拉格朗日形式
\begin{equation}\label{key}
(p_{1}^{\lambda}):\qquad \min_{x} \dfrac{1}{2}\Vert y-Ax\Vert_2^2 + \lambda \Vert x\Vert_1 \qquad
\end{equation}

其中$ x $为$ n $维未知稀疏信号,从统计学角度来说对应于线性回归中的稀疏向量,每一个系数$ x_i $表示第$ i $个输入或预测因子$ A_i $对输出$ y $的影响程度,$ y $为目标变量$ Y $的$ m $维观测向量。$ A $为$ m\times n $维设计矩阵,其第$ i $列为随机变量$ A_i $的$ m $维样本,即$ A $为$ m $个独立同分布的观测集合。


在讨论稀疏复原问题的特定算法之前,需要指出问题$(p_{1}^{\epsilon})  $和问题$ (p_{1}^{\lambda}) $可以利用一般的优化技术求解。然而在实际应用中,收敛速度可能比较慢,而且解通常是非稀疏的。

而且问题$ (p_{1}^{\lambda}) $可以转化为一个二次规划问题,从而可以应用一般的工具箱,如CVX对其进行求解。但该方法适用于小规模问题。一般的二次规划问题求解复杂度并不与问题规模成正比。因此,{\heiti 有必要挖掘稀疏复原问题的特定结构,研究求解问题$(p_{1}^{\epsilon})  $和问题$ (p_{1}^{\lambda}) $的特定方法,并将问题扩展到其他类型的目标函数与正则函数。}

\section{一元阈值是正交设计的最优方法}
在开始介绍求解上述问题的方法前,将先考虑正交设计矩阵的特殊情况。事实证明,当$\ell_0$ 与$\ell_1$范数优化问题分解为独立的一元问题时,其最优解可以通过非常简单的一元阈值过程得到。

$ n $维正交标准矩阵满足$ A^T A =AA^T=I$,由正交矩阵$ A $定义的线性变换具有一个良好的性质:保持向量的$ \ell_2 $范数不变,即:
$$  \Vert Ax \Vert_2^2 = (Ax)^T(Ax) =x^T(A^TA)x = x^Tx = \Vert x \Vert_2^2$$

对于$  $

\section{求解$ \ell_0 $范数最小化的算法}




\section{用于$ \ell_1 $范数最小化的算法}